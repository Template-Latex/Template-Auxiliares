% Template:     Auxiliar LaTeX
% Documento:    Configuración de página
% Versión:      8.0.8 (10/01/2022)
% Codificación: UTF-8
%
% Autor: Pablo Pizarro R.
%        pablo@ppizarror.com
%
% Manual template: [https://latex.ppizarror.com/auxiliares]
% Licencia MIT:    [https://opensource.org/licenses/MIT]

\newcommand{\templatePagecfg}{
	
	% -------------------------------------------------------------------------
	% Numeración de páginas
	% -----------------------------------------------------------------------------
	\clearpage
	\renewcommand{\thepage}{\arabic{page}}
	\setcounter{page}{1}
	\setcounter{section}{0}
	\setcounter{footnote}{0}
	\ifthenelse{\equal{\showsectioncaptioncode}{none}}{ % Numeración de la sección en los objetos CÓDIGO FUENTE
		\def\sectionobjectnumcode {}
	}{
	\ifthenelse{\equal{\showsectioncaptioncode}{sec}}{
		\def\sectionobjectnumcode {\thesection\sectioncaptiondelimiter}
	}{
	\ifthenelse{\equal{\showsectioncaptioncode}{ssec}}{
		\def\sectionobjectnumcode {\thesubsection\sectioncaptiondelimiter}
	}{
	\ifthenelse{\equal{\showsectioncaptioncode}{sssec}}{
		\def\sectionobjectnumcode {\thesubsubsection\sectioncaptiondelimiter}
	}{
	\ifthenelse{\equal{\showsectioncaptioncode}{ssssec}}{
		\def\sectionobjectnumcode {\thesubsubsubsection\sectioncaptiondelimiter}
	}{
	\ifthenelse{\equal{\showsectioncaptioncode}{chap}}{
		\def\sectionobjectnumcode {\thechapter\sectioncaptiondelimiter}
	}{
		\throwbadconfig{Valor configuracion incorrecto}{\showsectioncaptioncode}{none,chap,sec,ssec,sssec,ssssec}
	}}}}}}
	\ifthenelse{\equal{\showsectioncaptioneqn}{none}}{ % Numeración de la sección en los objetos ECUACIONES
		\def\sectionobjectnumeqn {}
	}{
	\ifthenelse{\equal{\showsectioncaptioneqn}{sec}}{
		\def\sectionobjectnumeqn {\thesection\sectioncaptiondelimiter}
	}{
	\ifthenelse{\equal{\showsectioncaptioneqn}{ssec}}{
		\def\sectionobjectnumeqn {\thesubsection\sectioncaptiondelimiter}
	}{
	\ifthenelse{\equal{\showsectioncaptioneqn}{sssec}}{
		\def\sectionobjectnumeqn {\thesubsubsection\sectioncaptiondelimiter}
	}{
	\ifthenelse{\equal{\showsectioncaptioneqn}{ssssec}}{
		\def\sectionobjectnumeqn {\thesubsubsubsection\sectioncaptiondelimiter}
	}{
	\ifthenelse{\equal{\showsectioncaptioneqn}{chap}}{
		\def\sectionobjectnumeqn {\thechapter\sectioncaptiondelimiter}
	}{
		\throwbadconfig{Valor configuracion incorrecto}{\showsectioncaptioneqn}{none,chap,sec,ssec,sssec,ssssec}
	}}}}}}
	\ifthenelse{\equal{\showsectioncaptionfig}{none}}{ % Numeración de la sección en los objetos FIGURAS
		\def\sectionobjectnumfig {}
	}{
	\ifthenelse{\equal{\showsectioncaptionfig}{sec}}{
		\def\sectionobjectnumfig {\thesection\sectioncaptiondelimiter}
	}{
	\ifthenelse{\equal{\showsectioncaptionfig}{ssec}}{
		\def\sectionobjectnumfig {\thesubsection\sectioncaptiondelimiter}
	}{
	\ifthenelse{\equal{\showsectioncaptionfig}{sssec}}{
		\def\sectionobjectnumfig {\thesubsubsection\sectioncaptiondelimiter}
	}{
	\ifthenelse{\equal{\showsectioncaptionfig}{ssssec}}{
		\def\sectionobjectnumfig {\thesubsubsubsection\sectioncaptiondelimiter}
	}{
	\ifthenelse{\equal{\showsectioncaptionfig}{chap}}{
		\def\sectionobjectnumfig {\thechapter\sectioncaptiondelimiter}
	}{
		\throwbadconfig{Valor configuracion incorrecto}{\showsectioncaptionfig}{none,chap,sec,ssec,sssec,ssssec}
	}}}}}}
	\ifthenelse{\equal{\showsectioncaptiontab}{none}}{ % Numeración de la sección en los objetos TABLAS
		\def\sectionobjectnumtab {}
	}{
	\ifthenelse{\equal{\showsectioncaptiontab}{sec}}{
		\def\sectionobjectnumtab {\thesection\sectioncaptiondelimiter}
	}{
	\ifthenelse{\equal{\showsectioncaptiontab}{ssec}}{
		\def\sectionobjectnumtab {\thesubsection\sectioncaptiondelimiter}
	}{
	\ifthenelse{\equal{\showsectioncaptiontab}{sssec}}{
		\def\sectionobjectnumtab {\thesubsubsection\sectioncaptiondelimiter}
	}{
	\ifthenelse{\equal{\showsectioncaptiontab}{ssssec}}{
		\def\sectionobjectnumtab {\thesubsubsubsection\sectioncaptiondelimiter}
	}{
	\ifthenelse{\equal{\showsectioncaptiontab}{chap}}{
		\def\sectionobjectnumtab {\thechapter\sectioncaptiondelimiter}
	}{
		\throwbadconfig{Valor configuracion incorrecto}{\showsectioncaptiontab}{none,chap,sec,ssec,sssec,ssssec}
	}}}}}}
	\ifthenelse{\equal{\captionnumcode}{arabic}}{ % Código fuente, INCLUIR SECCIÓN
		\renewcommand{\thelstlisting}{\sectionobjectnumcode\arabic{lstlisting}}
	}{
	\ifthenelse{\equal{\captionnumcode}{alph}}{
		\renewcommand{\thelstlisting}{\sectionobjectnumcode\alph{lstlisting}}
	}{
	\ifthenelse{\equal{\captionnumcode}{Alph}}{
		\renewcommand{\thelstlisting}{\sectionobjectnumcode\Alph{lstlisting}}
	}{
	\ifthenelse{\equal{\captionnumcode}{roman}}{
		\renewcommand{\thelstlisting}{\sectionobjectnumcode\roman{lstlisting}}
	}{
	\ifthenelse{\equal{\captionnumcode}{Roman}}{
		\renewcommand{\thelstlisting}{\sectionobjectnumcode\Roman{lstlisting}}
	}{
		\throwbadconfig{Tipo numero codigo fuente desconocido}{\captionnumcode}{arabic,alph,Alph,roman,Roman}}}}}
	}
	\ifthenelse{\equal{\captionnumequation}{arabic}}{ % Ecuaciones, INCLUIR SECCIÓN
		\renewcommand{\theequation}{\sectionobjectnumeqn\arabic{equation}}
	}{
	\ifthenelse{\equal{\captionnumequation}{alph}}{
		\renewcommand{\theequation}{\sectionobjectnumeqn\alph{equation}}
	}{
	\ifthenelse{\equal{\captionnumequation}{Alph}}{
		\renewcommand{\theequation}{\sectionobjectnumeqn\Alph{equation}}
	}{
	\ifthenelse{\equal{\captionnumequation}{roman}}{
		\renewcommand{\theequation}{\sectionobjectnumeqn\roman{equation}}
	}{
	\ifthenelse{\equal{\captionnumequation}{Roman}}{
		\renewcommand{\theequation}{\sectionobjectnumeqn\Roman{equation}}
	}{
		\throwbadconfig{Tipo numero ecuacion desconocido}{\captionnumequation}{arabic,alph,Alph,roman,Roman}}}}}
	}
	\ifthenelse{\equal{\captionnumfigure}{arabic}}{ % Figuras, INCLUIR SECCIÓN
		\renewcommand{\thefigure}{\sectionobjectnumfig\arabic{figure}}
	}{
	\ifthenelse{\equal{\captionnumfigure}{alph}}{
		\renewcommand{\thefigure}{\sectionobjectnumfig\alph{figure}}
	}{
	\ifthenelse{\equal{\captionnumfigure}{Alph}}{
		\renewcommand{\thefigure}{\sectionobjectnumfig\Alph{figure}}
	}{
	\ifthenelse{\equal{\captionnumfigure}{roman}}{
		\renewcommand{\thefigure}{\sectionobjectnumfig\roman{figure}}
	}{
	\ifthenelse{\equal{\captionnumfigure}{Roman}}{
		\renewcommand{\thefigure}{\sectionobjectnumfig\Roman{figure}}
	}{
		\throwbadconfig{Tipo numero figura desconocido}{\captionnumfigure}{arabic,alph,Alph,roman,Roman}}}}}
	}
	\ifthenelse{\equal{\captionnumsubfigure}{arabic}}{ % Subfiguras, NO USAR SECCIONES YA QUE SON HIJAS DE FIGURA
		\renewcommand{\thesubfigure}{\arabic{subfigure}}
	}{
	\ifthenelse{\equal{\captionnumsubfigure}{alph}}{
		\renewcommand{\thesubfigure}{\alph{subfigure}}
	}{
	\ifthenelse{\equal{\captionnumsubfigure}{Alph}}{
		\renewcommand{\thesubfigure}{\Alph{subfigure}}
	}{
	\ifthenelse{\equal{\captionnumsubfigure}{roman}}{
		\renewcommand{\thesubfigure}{\roman{subfigure}}
	}{
	\ifthenelse{\equal{\captionnumsubfigure}{Roman}}{
		\renewcommand{\thesubfigure}{\Roman{subfigure}}
	}{
		\throwbadconfig{Tipo numero subfigura desconocido}{\captionnumsubfigure}{arabic,alph,Alph,roman,Roman}}}}}
	}
	\ifthenelse{\equal{\captionnumtable}{arabic}}{ % Tablas, INCLUIR SECCIÓN
		\renewcommand{\thetable}{\sectionobjectnumtab\arabic{table}}
	}{
	\ifthenelse{\equal{\captionnumtable}{alph}}{
		\renewcommand{\thetable}{\sectionobjectnumtab\alph{table}}
	}{
	\ifthenelse{\equal{\captionnumtable}{Alph}}{
		\renewcommand{\thetable}{\sectionobjectnumtab\Alph{table}}
	}{
	\ifthenelse{\equal{\captionnumtable}{roman}}{
		\renewcommand{\thetable}{\sectionobjectnumtab\roman{table}}
	}{
	\ifthenelse{\equal{\captionnumtable}{Roman}}{
		\renewcommand{\thetable}{\sectionobjectnumtab\Roman{table}}
	}{
		\throwbadconfig{Tipo numero tabla desconocido}{\captionnumtable}{arabic,alph,Alph,roman,Roman}}}}}
	}
	\ifthenelse{\equal{\captionnumsubtable}{arabic}}{ % Subtablas, NO INCLUIR SECCIÓN YA QUE SON HIJAS DE LAS TABLAS
		\renewcommand{\thesubtable}{\arabic{subtable}}
	}{
	\ifthenelse{\equal{\captionnumsubtable}{alph}}{
		\renewcommand{\thesubtable}{\alph{subtable}}
	}{
	\ifthenelse{\equal{\captionnumsubtable}{Alph}}{
		\renewcommand{\thesubtable}{\Alph{subtable}}
	}{
	\ifthenelse{\equal{\captionnumsubtable}{roman}}{
		\renewcommand{\thesubtable}{\roman{subtable}}
	}{
	\ifthenelse{\equal{\captionnumsubtable}{Roman}}{
		\renewcommand{\thesubtable}{\Roman{subtable}}
	}{
		\throwbadconfig{Tipo numero subtabla desconocido}{\captionnumsubtable}{arabic,alph,Alph,roman,Roman}}}}}
	}

	% -------------------------------------------------------------------------
	% Márgenes de páginas y tablas
	% -----------------------------------------------------------------------------
	\setpagemargincm{\pagemarginleft}{\pagemargintop}{\pagemarginright}{\pagemarginbottom}
	\resettablecellpadding

	% -------------------------------------------------------------------------
	% Se define el punto decimal
	% -------------------------------------------------------------------------
	\ifthenelse{\equal{\pointdecimal}{true}}{
		\decimalpoint}{
	}
	
	% -------------------------------------------------------------------------
	% Definición de nombres de objetos
	% -------------------------------------------------------------------------
	\renewcommand{\abstractname}{\nameabstract} % Nombre del abstract
	\renewcommand{\appendixname}{\nameltappendixsection} % Nombre del anexo (título)
	\renewcommand{\appendixpagename}{\nameappendixsection} % Nombre del anexo en índice
	\renewcommand{\appendixtocname}{\nameappendixsection} % Nombre del anexo en índice
	\renewcommand{\contentsname}{\nameltcont} % Nombre del índice
	\renewcommand{\figurename}{\nameltwfigure} % Nombre de la leyenda de las fig.
	\renewcommand{\listfigurename}{\nameltfigure} % Nombre del índice de figuras
	\renewcommand{\listtablename}{\namelttable} % Nombre del índice de tablas
	\renewcommand{\lstlistingname}{\nameltwsrc} % Nombre leyenda del código fuente
	\renewcommand{\lstlistlistingname}{\nameltsrc} % Nombre índice código fuente
	\renewcommand{\refname}{\namereferences} % Nombre de las referencias (bibtex)
	\renewcommand{\bibname}{\namereferences} % Nombre de las referencias (natbib)
	\renewcommand{\tablename}{\nameltwtable} % Nombre de la leyenda de tablas
	
	% -------------------------------------------------------------------------
	% Estilo de títulos
	% -------------------------------------------------------------------------
	\sectionfont{%
		\color{\sectioncolor} \sectionfontsize \sectionfontstyle \selectfont%
	}
	\subsectionfont{%
		\color{\ssectioncolor} \ssectionfontsize \ssectionfontstyle \selectfont%
	}
	\subsubsectionfont{%
		\color{\sssectioncolor} \sssectionfontsize \sssectionfontstyle \selectfont%
	}
	\titleformat{\subsubsubsection}{%
		\color{\ssssectioncolor} \ssssectionfontsz \ssssectionfontstyle%
	}{%
		\thesubsubsubsection\charaftersectionnum\spacingaftersection%
		\corepatchaftersubsubsubsection%
	}{0em}{%
	}
	\def\bibfont {\fontsizerefbibl} % Tamaño de fuente de las referencias
	
	% -------------------------------------------------------------------------
	% Estilo citas
	% -------------------------------------------------------------------------
	\ifthenelse{\equal{\stylecitereferences}{apacite}}{%
		\renewcommand{\BOthers}[1]{\apacitebothers\hbox{}}%
	}{}
	
	% -------------------------------------------------------------------------
	% Se crean los header-footer
	% -----------------------------------------------------------------------------
	\fancyheadoffset{0pt}
	\ifthenelse{\equal{\templatestyle}{style1}}{
		\pagestyle{fancy}
		\newcommand{\COREstyledefinition}{
			\fancyhf{} % Headers y footers
			\fancyhead[L]{}
			\fancyhead[R]{}
			\renewcommand{\headrulewidth}{0pt} % Ancho de la barra del header
			\fancyfoot[L]{\small \rm \textit{\documenttitle}} % Footer izq, título del informe
			\fancyfoot[R]{\small \rm \nouppercase{\thepage}} % Footer der, curso
			\renewcommand{\footrulewidth}{0.5pt} % Ancho de la barra del footer
		}
		\COREstyledefinition
		\fancypagestyle{styleportrait}{ % Estilo portada
			\pagestyle{fancy}
			\fancyhf{}
			\renewcommand{\headrulewidth}{0pt}
			%\setpagemargincm{\pagemarginleft}{\pagemargintop}{\pagemarginright}{\pagemarginbottom}
			\fancyfoot[L]{\small \rm \textit{\documenttitle}} % Footer izq, título
			\fancyfoot[R]{\small \rm \nouppercase{\thepage}} % Footer der, curso
			\renewcommand{\footrulewidth}{0.5pt} % Ancho de la barra del footer
		}
		\thispagestyle{styleportrait} % Encabezado auxiliar (título e integrantes)
		\begin{spacing}{1.025}
		\begin{flushleft} % Crea el header
			\hspace{0cm}
			\vspace{-0cm}
			\begin{tabular}{c}
				\hspace{-0.45cm}~
				\begin{minipage}[t]{1\linewidth}
					\vspace{-4.65em}
					\universityname ~ \\
					\universityfaculty ~ \\
					\universitydepartment ~ \\
					\coursecode\ \hfpdashcharstyle\ \coursename ~ \\
					\begin{flushright}%
						\vspace{-6.7em}\nobreak~\coreinsertkeyimage{\universitydepartmentimagecfg}{\universitydepartmentimage} \vspace{0cm} ~
					\end{flushright}
				\end{minipage}
				\\
			\end{tabular}
			~
		\end{flushleft}
		\end{spacing}
		\vspace*{-1.15cm}
		\noindent \rule{1\linewidth}{0.5pt}
		\begin{center} % Tabla de datos
			\vspace*{0.35cm}
			\huge {\documenttitle} ~ \\
			\ifthenelse{\equal{\documentsubject}{\xspace}}{}{
				\vspace{0.2cm}
				\normalsize {\documentsubject} ~ \\
			}
			\ifdefempty{\teachingstaff}{\vspace{0.4cm}}{
				\vspace{0.5cm}
				\begin{spacing}{0.95}
					\begin{normalsize}
						\teachingstaff
					\end{normalsize}
				\end{spacing}
				\vspace{0.4cm}
			}
		\end{center}
	}{
	\ifthenelse{\equal{\templatestyle}{style2}}{
		\pagestyle{fancy}
		\newcommand{\COREstyledefinition}{
			\fancyhf{} % Headers y footers
			\fancyhead[L]{}
			\fancyhead[R]{}
			\renewcommand{\headrulewidth}{0pt} % Ancho de la barra del header
			\fancyfoot[L]{} % Footer izq
			\fancyfoot[C]{\thepage} % Footer centro
			\fancyfoot[R]{} % Footer der
			\renewcommand{\footrulewidth}{0pt} % Ancho de la barra del footer
		}
		\COREstyledefinition
		\fancypagestyle{styleportrait}{ % Estilo portada
			\pagestyle{fancy}
			\fancyhf{}
			\renewcommand{\headrulewidth}{0pt}
			%\setpagemargincm{\pagemarginleft}{\pagemargintop}{\pagemarginright}{\pagemarginbottom}
			\fancyfoot[L]{} % Footer izq
			\fancyfoot[C]{\thepage} % Footer centro
			\fancyfoot[R]{} % Footer der
			\renewcommand{\footrulewidth}{0pt} % Ancho de la barra del footer
		}
		\thispagestyle{styleportrait} % Encabezado auxiliar (título e integrantes)
		\begin{spacing}{1.025}
		\begin{flushleft}
			\hspace{0cm}
			\vspace{-0cm}
			\begin{tabular}{c}
				\hspace{-0.60cm}
				\begin{minipage}[t]{1.0\linewidth}
					\begin{flushleft}
						\vspace{-5.1em}\nobreak~\coreinsertkeyimage{\universitydepartmentimagecfg}{\universitydepartmentimage} \vspace{0cm}
					\end{flushleft}
				\end{minipage}
				\\
			\end{tabular}
			~
		\end{flushleft}
		\end{spacing}
		\vspace*{-1.25cm}
		\begin{center} % Tabla de datos
			\vspace*{0.35cm}
			\LARGE {\textbf{\documenttitle}} ~ \\
			\vspace{0.3cm}
			\large {\coursecode \coursename} ~ \\
			\ifthenelse{\equal{\documentsubject}{\xspace}}{}{
				\vspace{0.1cm}
				\normalsize {\documentsubject} ~ \\
			}
			\ifdefempty{\teachingstaff}{\vspace{0.4cm}}{
				\vspace{0.5cm}
				\begin{spacing}{0.95}
					\begin{normalsize}
						\teachingstaff
					\end{normalsize}
				\end{spacing}
				\vspace{0.4cm}
			}
		\end{center}
	}{
	\ifthenelse{\equal{\templatestyle}{style3}}{
		\pagestyle{fancy}
		\newcommand{\COREstyledefinition}{
			\fancyhf{} % Headers y footers
			\fancyhead[L]{}
			\fancyhead[R]{}
			\renewcommand{\headrulewidth}{0pt} % Ancho de la barra del header
			\fancyfoot[L]{} % Footer izq
			\fancyfoot[C]{\thepage} % Footer centro
			\fancyfoot[R]{} % Footer der
			\renewcommand{\footrulewidth}{0pt} % Ancho de la barra del footer
		}
		\COREstyledefinition
		\fancypagestyle{styleportrait}{ % Estilo portada
			\pagestyle{fancy}
			\fancyhf{}
			\renewcommand{\headrulewidth}{0pt}
			%\setpagemargincm{\pagemarginleft}{\pagemargintop}{\pagemarginright}{\pagemarginbottom}
			\fancyfoot[L]{} % Footer izq
			\fancyfoot[C]{\thepage} % Footer centro
			\fancyfoot[R]{} % Footer der
			\renewcommand{\footrulewidth}{0pt} % Ancho de la barra del footer
		}
		\thispagestyle{styleportrait} % Encabezado auxiliar (título e integrantes)
		\begin{spacing}{1.025}
		\begin{flushleft}
			\hspace{0cm}
			\vspace{-0cm}
			\begin{tabular}{c}
				\hspace{-0.33cm}
				\begin{minipage}[t]{1.0\linewidth}
					\begin{flushright}
						\vspace{-5.1em}\nobreak~\coreinsertkeyimage{\universitydepartmentimagecfg}{\universitydepartmentimage} \vspace{0cm}
					\end{flushright}
				\end{minipage}
				\\
			\end{tabular}
			~
		\end{flushleft}
		\end{spacing}
		\vspace*{-1.25cm}
		\begin{center} % Tabla de datos
			\vspace*{0.35cm}
			\LARGE {\textbf{\documenttitle}} ~ \\
			\vspace{0.3cm}
			\large {\coursecode \coursename} ~ \\
			\ifthenelse{\equal{\documentsubject}{\xspace}}{}{
				\vspace{0.1cm}
				\normalsize {\documentsubject} ~ \\
			}
			\ifdefempty{\teachingstaff}{\vspace{0.4cm}}{
				\vspace{0.5cm}
				\begin{spacing}{0.95}
					\begin{normalsize}
						\teachingstaff
					\end{normalsize}
				\end{spacing}
				\vspace{0.4cm}
			}
		\end{center}
	}{
	\ifthenelse{\equal{\templatestyle}{style4}}{
		\pagestyle{fancy}
		\newcommand{\COREstyledefinition}{
			\fancyhf{} % Headers y footers
			\fancyhead[L]{}
			\fancyhead[R]{}
			\renewcommand{\headrulewidth}{0pt} % Ancho de la barra del header
			\fancyfoot[L]{\small \rm \textit{\documenttitle}} % Footer izq, título del informe
			\fancyfoot[R]{\small \rm \nouppercase{\thepage}} % Footer der, curso
			\renewcommand{\footrulewidth}{0.5pt} % Ancho de la barra del footer
		}
		\COREstyledefinition
		\fancypagestyle{styleportrait}{ % Estilo portada
			\pagestyle{fancy}
			\fancyhf{}
			\renewcommand{\headrulewidth}{0pt}
			%\setpagemargincm{\pagemarginleft}{\pagemargintop}{\pagemarginright}{\pagemarginbottom}
			\fancyfoot[L]{\small \rm \textit{\documenttitle}} % Footer izq, título
			\fancyfoot[R]{\small \rm \nouppercase{\thepage}} % Footer der, curso
			\renewcommand{\footrulewidth}{0.5pt} % Ancho de la barra del footer
		}
		\thispagestyle{styleportrait} % Encabezado auxiliar (título e integrantes)
		\begin{spacing}{1.025}
		\begin{flushleft} % Crea el header
			\hspace{0cm}
			\vspace{-0cm}
			\begin{tabular}{c}
				\hspace{-0.45cm}~
				\begin{minipage}[t]{1\linewidth}
					\vspace{-4.65em}
					\universityname ~ \\
					\universityfaculty ~ \\
					\universitydepartment ~ \\
					\coursecode\ \hfpdashcharstyle\ \coursename ~ \\
					\begin{flushright}%
						\vspace{-6.7em}\nobreak~\coreinsertkeyimage{\universitydepartmentimagecfg}{\universitydepartmentimage} \vspace{0cm} ~
					\end{flushright}
				\end{minipage}
				\\
			\end{tabular}
			~
		\end{flushleft}
		\end{spacing}
		\vspace*{-1.15cm}
		\noindent \rule{1\linewidth}{0.5pt}
	}{
		\throwbadconfigondoc{Estilo de template incorrecto}{\templatestyle}{style1,style2,style3,style4}}}}
	}
	% Aplica el estilo de página
	\fancypagestyle{plain}{
		\fancyheadoffset{0pt}
		\COREstyledefinition
	}
	% Define estilos por defecto en flotantes
	\floatpagestyle{plain}
	\rotfloatpagestyle{plain}

	% -------------------------------------------------------------------------
	% Muestra los números de línea
	% -------------------------------------------------------------------------
	\ifthenelse{\equal{\showlinenumbers}{true}}{
		\linenumbers}{
	}

	% -------------------------------------------------------------------------
	% Reestablece \cleardoublepage
	% -------------------------------------------------------------------------
	% \let\cleardoublepage\oldcleardoublepage
	\let\cleardoublepage\corecleardoublepage

	% -----------------------------------------------------------------------------
	% Establece el estilo de las sub-sub-sub-secciones
	% -----------------------------------------------------------------------------
	\titleclass{\subsubsubsection}{straight}[\subsection]

	% -------------------------------------------------------------------------
	% Reestablece los valores del estado de los títulos
	% -------------------------------------------------------------------------
	\global\def\GLOBALtitlerequirechapter {false}
	\global\def\GLOBALtitleinitchapter {false}
	\global\def\GLOBALtitleinitsection {false}
	\global\def\GLOBALtitleinitsubsection {false}
	\global\def\GLOBALtitleinitsubsubsection {false}
	\global\def\GLOBALtitleinitsubsubsubsection {false}

}
