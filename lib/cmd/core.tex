% Template:     Template Auxiliar LaTeX
% Documento:    Funciones del núcleo del template
% Versión:      5.0.0 (22/04/2018)
% Codificación: UTF-8
%
% Autor: Pablo Pizarro R. @ppizarror
%        Facultad de Ciencias Físicas y Matemáticas
%        Universidad de Chile
%        pablo.pizarro@ing.uchile.cl, ppizarror.com
%
% Sitio web:    [http://latex.ppizarror.com/Template-Auxiliares/]
% Licencia MIT: [https://opensource.org/licenses/MIT]

% Lanza un mensaje de error
% 	#1	Función del error
%	#2	Mensaje
\newcommand{\throwerror}[2]{
	\errmessage{LaTeX Error: \noexpand#1 #2 (linea \the\inputlineno)}
	\stop
}

% Lanza un mensaje de advertencia
%	#1	Mensaje
\newcommand{\throwwarning}[1]{
	\errmessage{LaTeX Warning: #1 (linea \the\inputlineno)}
}

	% Lanza un mensaje de error indicando mala configuración
%	#1	Mensaje de error
% 	#2	Configuración usada
%	#3	Valores esperados
\newcommand{\throwbadconfig}[3]{
	\errmessage{LaTeX Warning: #1 \noexpand #2=#2. Valores esperados: #3}
	\stop
}

% Lanza un mensaje de error indicando mala configuración dentro de begin{document}
%	#1	Mensaje de error
% 	#2	Configuración usada
%	#3	Valores esperados
\newcommand{\throwbadconfigondoc}[3]{
	\errmessage{#1 \noexpand #2=#2. Valores esperados: #3}
	\stop
}

% Comprueba si una variable está definida
%	#1	Variable
\newcommand{\checkvardefined}[1]{
	\ifthenelse{\isundefined{#1}}{
		\errmessage{LaTeX Warning: Variable \noexpand#1 no definida}
		\stop
	}{}
}

% Comprueba si una variable está definida
%	#1	Variable
%	#2	Mensaje
\newcommand{\checkextravarexist}[2]{
	\ifthenelse{\isundefined{#1}}{
		\errmessage{LaTeX Warning: Variable \noexpand#1 no definida}
		\ifx\hfuzz#2\hfuzz
			\errmessage{LaTeX Warning: Defina la variable en el bloque de INFORMACION DEL DOCUMENTO al comienzo del archivo principal del Template}
		\else
			\errmessage{LaTeX Warning: #2}
		\fi
	}{}
}

% Lanza un mensaje de error si una variable no ha sido definida
% 	#1	Función del error
%	#2	Variable
%	#3	Mensaje
\newcommand{\emptyvarerr}[3]{
	\ifx\hfuzz#2\hfuzz
		\errmessage{LaTeX Warning: \noexpand#1 #3 (linea \the\inputlineno)}
	\fi
}

% Cambiar el margen de los caption
% 	#1	Margen en centímetros
\newcommand{\setcaptionmargincm}[1]{
	\captionsetup{margin=#1cm}
}

% Cambia márgenes de las páginas [cm]
% 	#1	Margen izquierdo
%	#2	Margen superior
%	#3	Margen derecho
%	#4	Margen inferior
\newcommand{\setpagemargincm}[4]{
	\newgeometry{left=#1cm, top=#2cm, right=#3cm, bottom=#4cm}
}

% Cambia los márgenes del documento
%	#1 Margen izquierdo
%	#2 Margen derecho
\newcommand{\changemargin}[2]{
	\emptyvarerr{\changemargin}{#1}{Margen izquierdo no definido}
	\emptyvarerr{\changemargin}{#2}{Margen derecho no definido}
	\list{}{\rightmargin#2\leftmargin#1}\item[]
}
\let\endchangemargin=\endlist

% Imagen de prueba tikz
% Inicia el modo de página horizontal
\def\beginplandscape {
	\newgeometry{landscape,left=\pagemarginleft cm,top=\pagemargintop cm,right=\pagemarginright cm,bottom=\pagemarginbottom cm}
	\headwidth=\textwidth
}

% Termina el modo de página horizontal
\def\endplandscape {
	\restoregeometry
	\setpagemargincm{\pagemarginleft}{\pagemargintop}{\pagemarginright}{\pagemarginbottom}
	\headwidth=\textwidth
}

% Chequea que las funciones sólo puedan usarse en el entorno images
\newcommand{\checkonlyonenvimage}{
	\ifthenelse{\equal{\envimagesinitialized}{true}}{}{
		\throwwarning{Funciones \noexpand\addimage o \noexpand\addimageboxed no pueden usarse fuera del entorno \noexpand\images}
		\stop
	}
}

% Chequea que las funciones sólo puedan usarse fuera del entorno images
\newcommand{\checkoutsideenvimage}{
	\ifthenelse{\equal{\envimagesinitialized}{true}}{
		\throwwarning{Esta funcion solo puede usarse fuera del entorno \noexpand\images}
		\stop
	}{}
}

% Alerta función de inserción de imagen deprecada, utilizar entorno images
\newcommand{\alertdeprecatedcmdimage}[1]{
	\throwwarning{Funcion \noexpand#1 deprecada, se recomienda utilizar el entorno \noexpand\images en su lugar}
}

% Se importa la librería tikz
\newcommand{\coreimporttikz}{
	\ifthenelse{\equal{\importtikz}{false}}{
		\usepackage{tikz}
	}{}
}
