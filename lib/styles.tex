% Template:     Template auxiliar LaTeX
% Documento:    Definición de estilos
% Versión:      2.0-beta-5 (09/04/2017)
% Codificación: UTF-8
%
% Autor: Pablo Pizarro R.
%        Facultad de Ciencias Físicas y Matemáticas.
%        Universidad de Chile.
%        pablo.pizarro@ing.uchile.cl, ppizarror.com
%
% Sitio web del proyecto: [http://ppizarror.com/Template-Auxiliares/]
% Licencia: MIT           [https://opensource.org/licenses/MIT]

\definecolor{dkgreen}{rgb}{0,0.6,0}
\definecolor{gray}{rgb}{0.5,0.5,0.5}
\definecolor{mauve}{rgb}{0.58,0,0.82}
\definecolor{mygreen}{rgb}{0,0.6,0}
\definecolor{mygray}{rgb}{0.5,0.5,0.5}
\definecolor{mymauve}{rgb}{0.58,0,0.82}
\definecolor{codegreen}{rgb}{0,0.6,0}
\definecolor{codegray}{rgb}{0.5,0.5,0.5}
\definecolor{codepurple}{rgb}{0.58,0,0.82}
\definecolor{backcolour}{rgb}{0.95,0.95,0.92}
\SelectInputMappings{
	% Definición de acentos
	aacute={á},
	Ntilde={Ñ},
	Euro={€}
}
\lstdefinestyle{C}{
	% Estilo de lenguaje C
	language=C,
	numbers=left,
	stepnumber=1,
	numbersep=5pt,
	backgroundcolor=\color{white},
	showspaces=false,
	showstringspaces=false,
	showtabs=false,
	tabsize=2,
	captionpos=b,
	breaklines=true,
	breakatwhitespace=true,
  	title=\lstname}
\lstdefinestyle{Java}{
	% Estilo de lenguaje Java
	language=Java,
	aboveskip=3mm,
	belowskip=3mm,
	showstringspaces=false,
	columns=flexible,
	basicstyle={\small\ttfamily},
	numbers=left,
	numberstyle=\tiny\color{gray},
	keywordstyle=\color{blue},
	commentstyle=\color{dkgreen},
	stringstyle=\color{mauve},
	breaklines=true,
	breakatwhitespace=true,
	tabsize=3,
	backgroundcolor=\color{backcolour}}
\lstdefinestyle{Matlab}{
	% Estilo de lenguaje Matlab
	language=Matlab,
	breaklines=true,
	morekeywords={matlab2tikz},
	keywordstyle=\color{blue},
	morekeywords=[2]{1}, keywordstyle=[2]{\color{black}},
	backgroundcolor=\color{backcolour},
	identifierstyle=\color{black},
	stringstyle=\color{mylilas},
	commentstyle=\color{mygreen},
	showstringspaces=false,
	numbers=left,
	showstringspaces=false,
	numberstyle={\tiny \color{black}},
	numbersep=9pt,
	basicstyle={\small\ttfamily},
	tabsize=3,
	breaklines=true,
	aboveskip=3mm,
	belowskip=3mm,
	emph=[1]{for,end,break},emphstyle=[1]\color{red}}
\lstdefinestyle{Python}{
	% Estilo de lenguaje Python
	language=Python,
	backgroundcolor=\color{backcolour},
	commentstyle=\color{codegreen},
	keywordstyle=\color{magenta},
	numberstyle=\tiny\color{codegray},
	stringstyle=\color{codepurple},
	basicstyle=\footnotesize,
	breakatwhitespace=false,
	breaklines=true,
	captionpos=b,
	keepspaces=true,
	numbers=left,
	numbersep=5pt,
	showspaces=false,
	showstringspaces=false,
	showtabs=false,
	tabsize=3,
	basicstyle={\small\ttfamily}
}
\lstset{literate=
	{á}{{\'a}}1 {é}{{\'e}}1 {í}{{\'i}}1 {ó}{{\'o}}1 {ú}{{\'u}}1
	{Á}{{\'A}}1 {É}{{\'E}}1 {Í}{{\'I}}1 {Ó}{{\'O}}1 {Ú}{{\'U}}1
	{à}{{\`a}}1 {è}{{\`e}}1 {ì}{{\`i}}1 {ò}{{\`o}}1 {ù}{{\`u}}1
	{À}{{\`A}}1 {È}{{\'E}}1 {Ì}{{\`I}}1 {Ò}{{\`O}}1 {Ù}{{\`U}}1
	{ä}{{\"a}}1 {ë}{{\"e}}1 {ï}{{\"i}}1 {ö}{{\"o}}1 {ü}{{\"u}}1
	{Ä}{{\"A}}1 {Ë}{{\"E}}1 {Ï}{{\"I}}1 {Ö}{{\"O}}1 {Ü}{{\"U}}1
	{â}{{\^a}}1 {ê}{{\^e}}1 {î}{{\^i}}1 {ô}{{\^o}}1 {û}{{\^u}}1
	{Â}{{\^A}}1 {Ê}{{\^E}}1 {Î}{{\^I}}1 {Ô}{{\^O}}1 {Û}{{\^U}}1
	{œ}{{\oe}}1 {Œ}{{\OE}}1 {æ}{{\ae}}1 {Æ}{{\AE}}1 {ß}{{\ss}}1
	{ű}{{\H{u}}}1 {Ű}{{\H{U}}}1 {ő}{{\H{o}}}1 {Ő}{{\H{O}}}1
	{ç}{{\c c}}1 {Ç}{{\c C}}1 {ø}{{\o}}1 {å}{{\r a}}1 {Å}{{\r A}}1
	{€}{{\EUR}}1 {£}{{\pounds}}1
}
\newcolumntype{P}[1]{>{\centering\arraybackslash}p{#1}} % Columna centrada en tablas
\fancypagestyle{styleportrait}{ % Estilo portada
	\pagestyle{fancy}
	\fancyhf{}
	\fancyhead[L]{
		\vspace*{3cm}
		\nombreuniversidad \\
		\nombrefacultad \\
		\departamentouniversidad \\
		\codigodelcurso - \nombredelcurso
	}
	\fancyhead[R]{
		\includegraphics[scale=\imagendeldepartamentoescl]{\imagendeldepartamento} \vspace{0cm}
	}
	\renewcommand{\headrulewidth}{0.5pt}
	\setpagemargincm{\defaultpagemarginleft}{2.0}
	{\defaultpagemarginright}{\defaultpagemarginbottom}
	\fancyfoot[L]{\small \rm \textit{\tituloauxiliar}} % Footer izq, título del informe
	\fancyfoot[R]{\small \rm \nouppercase{\thepage}}   % Footer der, curso
	\fancyfoot[L]{\small \rm \textit{\tituloauxiliar}} % Footer izq, título del informe
	\fancyfoot[R]{\small \rm \nouppercase{\thepage}}   % Footer der, curso
	\renewcommand{\footrulewidth}{0.5pt}               % Ancho de la barra del footer
}