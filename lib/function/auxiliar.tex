% Template:     Template Auxiliar LaTeX
% Documento:    Funciones exclusivas de Template-Auxiliar
% Versión:      4.2.9 (27/08/2017)
% Codificación: UTF-8
%
% Autor: Pablo Pizarro R.
%        Facultad de Ciencias Físicas y Matemáticas
%        Universidad de Chile
%        pablo.pizarro@ing.uchile.cl, ppizarror.com
%
% Sitio web:    [http://latex.ppizarror.com/Template-Auxiliares/]
% Licencia MIT: [https://opensource.org/licenses/MIT]

\newcommand{\newquestion}[1]{
	% Insertar nuevo título de pregunta
	%	#1	Título
	\emptyvarerr{\newquestion}{#1}{Titulo pregunta no definido}
	\sectionanum{#1}
}

\newcommand{\newboxquestion}[1]{
	% Insertar nuevo título de pregunta encerrado en un recuadro
	%	#1	Número pregunta
	\emptyvarerr{\newquestion}{#1}{Titulo pregunta no definido}
	\phantomsection
	\newp \fbox{\ \textbf{#1}.-\ } \noindent
	\pdfbookmark[1]{#1}{toc}
}

% Crea una sección de referencias solo para bibtex
\newenvironment{references}{
	\ifthenelse{\equal{\stylecitereferences}{bibtex}}{
	}{
		\throwerror{}{Solo se puede usar entorno references con estilo citas \noexpand\stylecitereferences=bibtex}
	}
	\begingroup
	% Se configura las referencias como una sección
	\ifthenelse{\equal{\donumrefsection}{true}}{
		\section{\namereferences}
	}{
		\sectionanum{\namereferences}
	}
	\renewcommand{\section}[2]{}
	\begin{thebibliography}{99}
	}
	{
	\end{thebibliography}
	\endgroup
}

% Crea una sección de anexos
\newenvironment{anexo}{
	\begingroup
	\clearpage
	\phantomsection
	\ifthenelse{\equal{\showappendixsectitle}{true}}{
		\appendixpage}{
	}
	\appendixtitleon
	\appendicestocpagenum
	\appendixtitletocon
	\bookmarksetup{
		numbered,
		openlevel=0
	}
	\begin{appendices}
		\bookmarksetupnext{level=part}
		\ifthenelse{\equal{\showappendixsecindex}{true}}{}{
			\belowpdfbookmark{\nameappendixsection}{contents}
		}
		\setcounter{secnumdepth}{3}
		\setcounter{tocdepth}{3}
		\ifthenelse{\equal{\appendixindepobjnum}{true}}{
			\counterwithin{equation}{section}
			\counterwithin{figure}{section}
			\counterwithin{lstlisting}{section}
			\counterwithin{table}{section}}{
		}
		}
		{
	\end{appendices}
	\bookmarksetupnext{level=0}
	\endgroup
}
