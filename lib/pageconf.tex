% Template:     Template auxiliar LaTeX
% Documento:    Configuración de página
% Versión:      3.0 (18/05/2017)
% Codificación: UTF-8
%
% Autor: Pablo Pizarro R.
%        Facultad de Ciencias Físicas y Matemáticas
%        Universidad de Chile
%        pablo.pizarro@ing.uchile.cl, ppizarror.com
%
% Sitio web del proyecto: [http://ppizarror.com/Template-Informe/]
% Licencia: MIT           [https://opensource.org/licenses/MIT]

% Numeración de páginas y márgenes
\newpage
\setpagemargincm{\pagemarginleft}{\pagemargintop}
{\pagemarginright}{\pagemarginbottom}
\decimalpoint % Se define el punto decimal
\def\arraystretch{\tablepadding} % Se ajusta el padding de las tablas

% Definición de nombres a objetos
\renewcommand{\sectionmark}[1]{\markboth{#1}{}} % Se modifica el estilo del header
\renewcommand{\tablename}{\nomltwtable}         % Nombre de la leyenda de tablas
\renewcommand{\figurename}{\nomltwfigure}       % Nombre de la leyenda de las fig.
\renewcommand{\lstlistingname}{\nomltwsrc}      % Nombre leyenda del código fuente

% Se crean los estílos de página
\pagestyle{fancy} \fancyhf{} \fancyhead[L]{} \fancyhead[R]{} % Headers y footers
\renewcommand{\headrulewidth}{0pt} % Ancho de la barra del header
\fancyfoot[L]{\small \rm \textit{\tituloauxiliar}} % Footer izq, título del informe
\fancyfoot[R]{\small \rm \nouppercase{\thepage}} % Footer der, curso
\renewcommand{\footrulewidth}{0.5pt} % Ancho de la barra del footer
\fancypagestyle{styleportrait}{ % Estilo portada
	\pagestyle{fancy}
	\fancyhf{}
	\fancyhead[L]{
		\vspace*{2.5cm}
		\nombreuniversidad \\
		\nombrefacultad \\
		\departamentouniversidad \\
		\codigodelcurso - \nombredelcurso
	}
	\fancyhead[R]{
		\includegraphics[scale=\imagendepartamentoescala]{\imagendepartamento} \vspace{0cm}
	}
	\renewcommand{\headrulewidth}{0.5pt}
	\setpagemargincm{\pagemarginleft}{2.0}
	{\pagemarginright}{\pagemarginbottom}
	\fancyfoot[L]{\small \rm \textit{\tituloauxiliar}} % Footer izq, título del informe
	\fancyfoot[R]{\small \rm \nouppercase{\thepage}}   % Footer der, curso
	\fancyfoot[L]{\small \rm \textit{\tituloauxiliar}} % Footer izq, título del informe
	\fancyfoot[R]{\small \rm \nouppercase{\thepage}}   % Footer der, curso
	\renewcommand{\footrulewidth}{0.5pt}               % Ancho de la barra del footer
}

% Encabezado auxiliar (título e integrantes)
\thispagestyle{styleportrait}
\begin{center}
	\vspace*{2cm}
	\huge {\tituloauxiliar} \\
	\vspace{0.2cm}
	\large {\temaatratar} \\
	\vspace{0.7cm}
	\equipodocente
	\vspace{0.6cm}
\end{center}

% Se activa el word-wrap para textos con \texttt{}
{{\ttfamily \hyphenchar\the\font=`\-}

% Se reestablece estilo de títulos
\sectionfont{\color{\titlecolor} \fontsizetitle \styletitle \selectfont}
\subsectionfont{\color{\subtitlecolor} \fontsizesubtitle \stylesubtitle \selectfont}
\subsubsectionfont{\color{\subsubtitlecolor} \fontsizesubsubtitle \stylesubsubtitle \selectfont}

% Se reestablecen números de página y secciones
\renewcommand{\thepage}{\arabic{page}}
\setcounter{page}{1}
\setcounter{section}{0}
\setcounter{footnote}{0}