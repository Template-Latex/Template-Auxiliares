% Template:     Template Auxiliar LaTeX
% Documento:    Configuración de página
% Versión:      4.2.5 (30/07/2017)
% Codificación: UTF-8
%
% Autor: Pablo Pizarro R.
%        Facultad de Ciencias Físicas y Matemáticas
%        Universidad de Chile
%        pablo.pizarro@ing.uchile.cl, ppizarror.com
%
% Sitio web:    [http://latex.ppizarror.com/Template-Auxiliares/]
% Licencia MIT: [https://opensource.org/licenses/MIT]

% Numeración de páginas
\newpage
\renewcommand{\thepage}{\arabic{page}}
\setcounter{page}{1}
\setcounter{section}{0}
\setcounter{footnote}{0}

% Márgenes de páginas y tablas
\setpagemargincm{\pagemarginleft}{\pagemargintop}{\pagemarginright}{\pagemarginbottom}
\def\arraystretch{\tablepadding} % Se ajusta el padding de las tablas

% Se define el punto decimal
\ifthenelse{\equal{\pointdecimal}{true}}{
	\decimalpoint}{
}

% Definición de nombres de objetos
\renewcommand{\appendixname}{\nomltappendixsection} % Nom. del anexo en etiq. de título
\renewcommand{\appendixpagename}{\nameappendixsection} % Nombre del anexo en índice
\renewcommand{\appendixtocname}{\nameappendixsection} % Nombre del anexo en índice
\renewcommand{\contentsname}{\nomltcont}  % Nombre del índice
\renewcommand{\figurename}{\nomltwfigure} % Nombre de la leyenda de las fig.
\renewcommand{\listfigurename}{\nomltfigure} % Nombre del índice de figuras
\renewcommand{\listtablename}{\nomlttable} % Nombre del índice de tablas
\renewcommand{\lstlistingname}{\nomltwsrc} % Nombre leyenda del código fuente
\renewcommand{\refname}{\namereferences} % Nombre de las referencias
\renewcommand{\tablename}{\nomltwtable} % Nombre de la leyenda de tablas

% Numeración de objetos
\ifthenelse{\equal{\showsectiononcaption}{true}}{
	\counterwithin{equation}{section}   % Añade número de sección a las ecuaciones
	\counterwithin{figure}{section}     % Añade número de sección a las figuras
	\counterwithin{lstlisting}{section} % Añade número de sección a los códigos
	\counterwithin{table}{section}      % Añade número de sección a las tablas
}{}

% Se crean los header-footer
\ifthenelse{\equal{\templatestyle}{style1}}{
	\pagestyle{fancy} \fancyhf{} \fancyhead[L]{} \fancyhead[R]{} % Headers y footers
	\renewcommand{\headrulewidth}{0pt} % Ancho de la barra del header
	\fancyfoot[L]{\small \rm \textit{\tituloauxiliar}} % Footer izq, título del informe
	\fancyfoot[R]{\small \rm \nouppercase{\thepage}} % Footer der, curso
	\renewcommand{\footrulewidth}{0.5pt} % Ancho de la barra del footer
	\fancypagestyle{styleportrait}{ % Estilo portada
		\pagestyle{fancy}
		\fancyhf{}
		\fancyhead[L]{
			\vspace*{2.5cm}
			\nombreuniversidad \\
			\nombrefacultad \\
			\departamentouniversidad \\
			\codigodelcurso\ - \nombredelcurso
		}
		\fancyhead[R]{
			\includegraphics[scale=\imagendepartamentoescala]{\imagendepartamento} \vspace{0cm}
		}
		\renewcommand{\headrulewidth}{0.5pt}
		\setpagemargincm{\pagemarginleft}{2.0}{\pagemarginright}{\pagemarginbottom}
		\fancyfoot[L]{\small \rm \textit{\tituloauxiliar}} % Footer izq, título del informe
		\fancyfoot[R]{\small \rm \nouppercase{\thepage}}   % Footer der, curso
		\renewcommand{\footrulewidth}{0.5pt}               % Ancho de la barra del footer
	}
	\thispagestyle{styleportrait} % Encabezado auxiliar (título e integrantes)
	\begin{center}
		\vspace*{2.0cm}
		\huge {\tituloauxiliar} \\
		\vspace{0.2cm}
		\large {\temaatratar} \\
		\ifx\hfuzz\equipodocente\hfuzz
		\else
			\vspace{0.7cm}
			\equipodocente
			\vspace{0.4cm}
		\fi
	\end{center}
}{
\ifthenelse{\equal{\templatestyle}{style2}}{
	\pagestyle{fancy} \fancyhf{} \fancyhead[L]{} \fancyhead[R]{} % Headers y footers
	\renewcommand{\headrulewidth}{0pt} % Ancho de la barra del header
	\fancyfoot[L]{} % Footer izq
	\fancyfoot[C]{\thepage} % Footer centro
	\fancyfoot[R]{} % Footer der
	\renewcommand{\footrulewidth}{0pt} % Ancho de la barra del footer
	\fancypagestyle{styleportrait}{ % Estilo portada
		\pagestyle{fancy}
		\fancyhf{}
		\fancyhead[L]{
			\includegraphics[scale=\imagendepartamentoescala]{\imagendepartamento} \vspace{0cm}
		}
		\fancyhead[R]{
			\vspace*{4.5cm}
		}
		\renewcommand{\headrulewidth}{0pt}
		\setpagemargincm{\pagemarginleft}{2.0}{\pagemarginright}{\pagemarginbottom}
		\fancyfoot[L]{} % Footer izq
		\fancyfoot[C]{\thepage} % Footer centro
		\fancyfoot[R]{} % Footer der
		\renewcommand{\footrulewidth}{0pt}               % Ancho de la barra del footer
	}
	\thispagestyle{styleportrait} % Encabezado auxiliar (título e integrantes)
	\begin{center}
		\vspace*{2.0cm}
		\LARGE {\textbf{\tituloauxiliar}} \\
		\vspace{0.3cm}
		\Large {\codigodelcurso \nombredelcurso} \\
		\vspace{0.2cm}
		\large {\temaatratar} \\
		\ifx\hfuzz\equipodocente\hfuzz
		\else
			\vspace{0.5cm}
			\equipodocente
			\vspace{0.4cm}
		\fi
	\end{center}
}{
	\throwbadconfigondoc{Estilo de template incorrecto}{\templatestyle}{style1,style2}}
}

% Tamaño fuentes
\sectionfont{\color{\titlecolor} \fontsizetitle \styletitle \selectfont}
\subsectionfont{\color{\subtitlecolor} \fontsizesubtitle \stylesubtitle \selectfont}
\subsubsectionfont{\color{\subsubtitlecolor} \fontsizesubsubtitle \stylesubsubtitle \selectfont}
