% Template:     Template auxiliar LaTeX
% Documento:    Configuración inicial del documento
% Versión:      2.0 (09/04/2017)
% Codificación: UTF-8
%
% Autor: Pablo Pizarro R.
%        Facultad de Ciencias Físicas y Matemáticas.
%        Universidad de Chile.
%        pablo.pizarro@ing.uchile.cl, ppizarror.com
%
% Sitio web del proyecto: [http://ppizarror.com/Template-Auxiliares/]
% Licencia: MIT           [https://opensource.org/licenses/MIT]

\decimalpoint                              % Se define el punto decimal
\setcaptionmargincm{\defaultcaptionmargin} % Margen por defecto
\setlength{\headheight}{64pt}              % Tamaño de la cabecera sin fancyhdr
\setcounter{MaxMatrixCols}{20}             % Número máximo de columnas para las matrices
\hypersetup{
	pdfauthor={\autordeldocumento},
	pdftitle={\tituloauxiliar},
	pdfsubject={\temaatratar},
	pdfkeywords={\nombreuniversidad, \nombredelcurso, \codigodelcurso},
	pdfcreator={pdfLaTeX, ppizarror},
	pdfproducer={Template LaTeX Auxiliar v\templateversion}}
\renewcommand{\baselinestretch}{\defaultinterlind} % Ajuste del entrelineado