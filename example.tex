% Template:     Template auxiliar LaTeX
% Documento:    Archivo de ejemplo
% Versión:      3.0 (18/05/2017)
% Codificación: UTF-8
%
% Autor: Pablo Pizarro R.
%        Facultad de Ciencias Físicas y Matemáticas.
%        Universidad de Chile.
%        pablo.pizarro@ing.uchile.cl, ppizarror.com
%
% Sitio web del proyecto: [http://ppizarror.com/Template-Auxiliares/]
% Licencia: MIT           [https://opensource.org/licenses/MIT]

% INTRODUCCIÓN AUXILIAR
\lipsum[11]

% PREGUNTA 1
\newquestion{Pregunta 1}

	% Párrafo
	\lipsum[2]
	
	\insertimage[\label{img:testimage}]{ejemplos/test-image.png}{scale=0.17}{Ubi sunt vobis? autem \quotes{Internet}}
	
	\newp Aequatio videbitur \footnote{ $Q_e = max\big(1 \dots n\big)$} infra:
	
	% Ecuación
	\insertequationalign[\label{eqn:eqn-larga}]{\lpow{\Lambda}{f} = \frac{L\cdot f}{W} \cdot \frac{\pow{\lpow{Q}{e}}{2}}{8 \pow{\pi}{2} \pow{W}{4} g}+ \sum_{i=1}^{l} \frac{f \cdot \big( M - d\big)}{l \cdot W} \cdot \frac{\pow{\big(\lpow{Q}{e}- i\cdot Q\big)}{2}}{8 \pow{\pi}{2} \pow{W}{4} g}}
	
% PREGUNTA 2
\newquestion{Pregunta 2}

	% Párrafo
	\lipsum[4]

	% Lista
	\begin{enumerate}
		\item {$Q = \omega \cdot \sum i$}
		\item {$e = i \pm \sqrt{1+k}$}
		\item {$K = \frac{1 + e}{1 - \delta}$}
	\end{enumerate}

% PREGUNTA 3
\newquestion{Pregunta 3}

	% Párrafo
	\lipsum[15] \newp

\lstset{style=Python}
\begin{lstlisting}
import numpy as np

def incmatrix(genl1,genl2):
m = len(genl1)
n = len(genl2)
M = None #to become the incidence matrix
VT = np.zeros((n*m,1), int)  #dummy variable
\end{lstlisting}