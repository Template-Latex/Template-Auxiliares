% Template:     Template Auxiliar LaTeX
% Documento:    Archivo de ejemplo
% Versión:      7.0.0 (15/06/2020)
% Codificación: UTF-8
%
% Autor: Pablo Pizarro R. @ppizarror
%        Facultad de Ciencias Físicas y Matemáticas.
%        Universidad de Chile.
%        pablo@ppizarror.com
%
% Sitio web:    [https://latex.ppizarror.com/auxiliares]
% Licencia MIT: [https://opensource.org/licenses/MIT]

% Introducción
\lipsum[11]

% Pregunta en forma de título
\newquestion{Pregunta 1}

	% Párrafo
	\lipsum[114]
	
	\insertimage[\label{img:testimage}]{ejemplos/test-image.png}{scale=0.17}{Ubi sunt vobis? autem \quotes{Internet}}
	
	\newp Aequatio videbitur\footnote{ $Q_e = max\big(1 \dots n\big)$} infra:
	
	% Ecuación
	\insertequation[\label{eqn:eqn-larga}]{\lpow{\Lambda}{f} = \frac{L\cdot f}{W} \cdot \frac{\pow{\lpow{Q}{e}}{2}}{8 \pow{\pi}{2} \pow{W}{4} g}+ \sum_{i=1}^{l} \frac{f \cdot \big( M - d\big)}{l \cdot W} \cdot \frac{\pow{\big(\lpow{Q}{e}- i\cdot Q\big)}{2}}{8 \pow{\pi}{2} \pow{W}{4} g}}

% Pregunta encerrada en un recuadro
\newpage
\newboxquestion{P2} \lipsum[4]

	% Lista
	\begin{enumerate}
		\item {$Q = \omega \cdot \sum i$}
		\item {$e = i \pm \sqrt{1+k}$}
		\item {$K = \frac{1 + e}{1 - \delta}$}
	\end{enumerate}

% Pregunta encerrada en un recuadro más control
\newboxquestion{P3} (\textbf{P1 Control 2 2017/1}) \lipsum[117] \newp 

\begin{sourcecode}{python}{}
import numpy as np

def incmatrix(genl1, genl2):
	m = len(genl1)
	n = len(genl2)
	M = None # Comentario 1
	VT = np.zeros((n*m, 1), int) # Comentario 2
\end{sourcecode}